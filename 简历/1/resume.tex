\documentclass{resume}
\usepackage{zh_CN-Adobefonts_external} 
\usepackage{linespacing_fix}
\usepackage{cite}
\begin{document}
\pagenumbering{gobble}



%***"%"后面的所有内容是注释而非代码,不会输出到最后的PDF中
%***使用本模板,只需要参照输出的PDF,在本文档的相应位置做简单替换即可
%***修改之后,输出更新后的PDF,只需要点击Overleaf中的“Recompile”按钮即可
%**********************************姓名********************************************
\name{xxx}
%**********************************联系信息****************************************
%第一个括号里写手机号,第二个写邮箱
\otherInfo{年龄:26岁}{意向城市:深圳}{}{}
\otherInfo{手机:xxx}{邮箱:xxx}{}{}
%**********************************其他信息****************************************
%在大括号内填写其他信息,最多填写4个,但是如果选择不填信息,
%那么大括号必须空着不写,而不能删除大括号。
%\otherInfo后面的四个大括号里的所有信息都会在一行输出
%如果想要写两行,那就用两次这个指令(\otherInfo{}{}{}{})即可

%*********************************照片**********************************************
%照片需要放到images文件夹下,名字必须是you.jpg,如果不需要照片可以不添加此行命令
%0.15的意思是,照片的宽度是页面宽度的0.15倍,调整大小,避免遮挡文字
\yourphoto{0.15}
%**********************************正文**********************************************


%***大标题,下面有横线做分割
%***一般的标题有:教育背景,实习(项目)经历,工作经历,自我评价,求职意向,等等

\section{教育背景}

%***********一行子标题**************
%***第一个大括号里的内容向左对齐,第二个大括号里的内容向右对齐
%***\textbf{}括号里的字是粗体,\textit{}括号里的字是斜体
\datedsubsection{\textbf{xxx} - xxx专业 - 本科}{2014.09 - 2018.06}

\section{专业技能}

\begin{itemize}[parsep=0.5ex]
  \item 熟练使用 xxx
  \item 熟练使用 xxx
  \item 熟练使用 xxx
  \item 熟悉xxx
  \item 熟练使用xxx
  \item 能够使用 xxx
  \item 了解 xxx
  \item 了解 xxx
\end{itemize}


\section{工作经历}

\datedsubsection{\textbf{xxx公司 - xxx工程师}}{2021.07 - 2023.03}
\begin{itemize}[parsep=0.5ex]
  \item 参与xxx
  \item 负责xxx
  \item 负责xxx
  \item 负责xxx
  \item 负责xxx
\end{itemize}

\datedsubsection{\textbf{xx公司 - xxx师}}{2019.04 - 2021.05}
\begin{itemize}[parsep=0.5ex]
  \item 负责xxx
  \item 利用xxx
  \item 负责xxx
\end{itemize}

\datedsubsection{\textbf{xxx有限公司 - xxx工程师}}{2018.07 - 2019.04}
\begin{itemize}[parsep=0.5ex]
  \item 负责xxx
  \item 负责xxx
  \item 负责xxx
\end{itemize}

\section{项目经历}

\datedsubsection{\textbf{xx项目}}{2021.07 - 2021.11}
\textbf{项目技术}:xxx\\
\textbf{项目描述}:xxx\\
\textbf{项目职责}:
\begin{enumerate}[parsep=0.5ex]
  \item xxx\textbf{加粗内容}xxx\textbf{加粗内容}
  \item xxx
  \item xxx\textbf{ xxx},xxx
  \item xxx
  \item xxx
  \item xxx
\end{enumerate}
\textbf{项目成果}:
\begin{enumerate}[parsep=0.5ex]
  \item xxx
  \item xxx
  \item xxx
\end{enumerate}


\datedsubsection{\textbf{xxx项目}}{2021.07 - 2021.08}
\textbf{项目技术}:xxx\\
\textbf{项目描述}:xxx\\
\textbf{项目职责}:
\begin{enumerate}[parsep=0.5ex]
  \item xxx
  \item xxxx
  \item xxx
  \item xxx
  \item xxx
  \item xxx
\end{enumerate}
\textbf{项目成果}:
\begin{enumerate}[parsep=0.5ex]
  \item xxx
  \item xxx
  \item xxx
\end{enumerate}
\textbf{}

\datedsubsection{\textbf{xxx项目}}{2021.07 - 2021.08}
\textbf{项目技术}:xxx\\
\textbf{项目描述}:xxx\\
\textbf{项目职责}:
\begin{enumerate}[parsep=0.5ex]
  \item xxx
  \item xxxx
  \item xxx
  \item xxx
  \item xxx
  \item xxx
\end{enumerate}
\textbf{项目成果}:
\begin{enumerate}[parsep=0.5ex]
  \item xxx
  \item xxx
  \item xxx
\end{enumerate}
\textbf{}

\datedsubsection{\textbf{xxx项目}}{2021.07 - 2021.08}
\textbf{项目技术}:xxx\\
\textbf{项目描述}:xxx\\
\textbf{项目职责}:
\begin{enumerate}[parsep=0.5ex]
  \item xxx
  \item xxxx
  \item xxx
  \item xxx
  \item xxx
  \item xxx
\end{enumerate}
\textbf{项目成果}:
\begin{enumerate}[parsep=0.5ex]
  \item xxx
  \item xxx
  \item xxx
\end{enumerate}
\textbf{}

\section{自我评价}

\begin{itemize}[parsep=0.5ex]
  \item xxx
  \item xxx
  \item xxx
  \item xxx
\end{itemize}

\end{document}
